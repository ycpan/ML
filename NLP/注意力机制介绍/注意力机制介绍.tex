\documentclass{ctexart}
\title{注意力机制及相关说明}
\author{DemonPan}
\date{2019-12-19}
\usepackage{indentfirst}
\begin{document}
\maketitle
当用神经网络来处理大量的输入信息时,也可以借鉴人脑的注意力机制,只 选择一些关键的信息输入进行处理,来提高神经网络的效率。
按照认知神经学中的注意力,可以总体上分为两类:\cite{ref1}\par
聚焦式(focus)注意力:自上而下的有意识的注意力,主动注意——是指有预定目的、依赖任务的、主动有意识地聚焦于某一对象的注意力;\par
显著性(saliency-based)注意力:自下而上的有意识的注意力,被动注意——基于显著性的注意力是由外界刺激驱动的注意,不需要主动干预,
也和任务无关;可以将max-pooling和门控(gating)机制来近似地看作是自下而上的基于显著性的注意力机制。\par
在人工神经网络中,注意力机制一般就特指聚焦式注意力。\par
注意力机制的理解。\par
在给定输入$\mathbf{X}={{\mathbf{x_1},\mathbf{x_2},\cdots,\mathbf{x_n}}}$的情况下,用$\mathbf{q}$表示当前任务,用$Attention(\mathbf{q},\mathbf{X})$表示$\mathbf{q}$对$\mathbf{X}$各元素的注意力,则
\begin{equation}Attention(\mathbf{q},\mathbf{X}) = softmax(score(\mathbf{q},\mathbf{X})) = [\alpha_1,\alpha_2,\cdots,\alpha_N]\end{equation}
$score$的计算方式有多种,如:
\begin{equation}score(\mathbf{q},\mathbf{x_i}) = \mathbf{q} \cdot \mathbf{x_i}\end{equation}\footnote{点积模型,点击模型和加法模型效果差不多,但点积模型可以
更好的利用矩阵的乘积,从而是计算效率更高}
\begin{equation}score(\mathbf{q},\mathbf{x_i}) = \frac{\mathbf{q} \cdot \mathbf{x_i}}{\sqrt{d_x}}\end{equation}\footnote{缩放点积模型:维度较大的点积会使得方差较大,从而导致softmax函数梯度比较小
不利于优化}
\begin{equation}score(\mathbf{q},\mathbf{x_i}) = \mathbf{W}\mathbf{q} + \mathbf{U}\mathbf{x_i}\end{equation}\footnote{加法模型:与点击模型效果区别不大}
用softmax求$score$的值,是为了让score值都为正值,并且使其分数之和为1的概率分布。\par
想象一种注意力场景。比如雪地上有一颗树,眼睛看到了 雪地也看到了树,但我们更多的注意到了树(当然雪地也看到了),可是眼睛只能在同一时间将一种信号送给大脑,那么这种信号该如何表示呢?\\
有了任务$\mathbf{q}$对$\mathbf{X}$不同的注意力,现在要做的是把对不同注意力的影响,融合成一种最终注意到的结果。用$r(\mathbf{X},\mathbf{q})$表示最终结果,则
%\begin{equation}r(\mathbf{X},\mathbf{q}) = \sum_{i=1}{N}softmax(score(\mathbf{x_i},\mathbf{q}))\\
%=\sum_{i=1}{i=N}\frac{e^score(mathbf{x_i},\mathbf{q})}{}\end{equation}
\begin{equation}r(\mathbf{X},\mathbf{q}) = \sum_{i=1}^{N}\alpha_i\mathbf{x_i}\end{equation}
%\renewcommand\refname{参考文献} 
\bibliography{注意力机制介绍}
\bibliographystyle{plain} 
\pagestyle{plain}
\end{document}